\chapter{Data Understanding and Preparation}
\label{ch:capitolo1}

% --- Inizio del Capitolo 1 ---

% Capitolo 1...

% Esempio di una nota\footnote{CleanCode} con citazione a sezione~\ref{sec:sezione1}.

% \begin{figure}[H]
%     \centering
%     \includegraphics[width=0.2\textwidth]{immagini/logo-dip_blu_hr.png}
%     \caption{Esempio di un'immagine}
%     \label{fig:immagine1}
% \end{figure}

% \begin{lstlisting}[caption={Esempio codice JavaScript},label={Esempio codice JavaScript}, language=JavaScript]
% function foo(num) {
%     const bar = 2;
%     return num + bar;
% }

% const result = foo(2);
% // result -> 4
% \end{lstlisting}


% \section{Sezione 1}\label{sec:sezione1}


% Sezione 1... con citazione bibliografica~\cite{CleanCode}

% \subsection{Sottosezione 1}
% \label{subsec:sottosezione1}

% Sottosezione 1...

\section{Data Introduction}\label{sec:data_intro}
% introduction to each column, how they store the data, etc...
% also should contain some graph to show the distributions etc...

% ['originalTitle', 'rating', 'startYear', 'endYear', 'runtimeMinutes',
% 'awardWins', 'numVotes', 'worstRating', 'bestRating', 'totalImages',
% 'totalVideos', 'totalCredits', 'criticReviewsTotal', 'titleType',
% 'awardNominationsExcludeWins', 'canHaveEpisodes', 'isRatable',
% 'isAdult', 'numRegions', 'userReviewsTotal', 'ratingCount',
% 'countryOfOrigin', 'genres']

% [Bruno: I think we can discuss correlations and eliminations here]
% e.g.:
% <variable1>
% <Paragraph about var1>
% 
% <variable2>, <variable3>
% Paragraph about var2, 3 and why they are correlated and can be eliminated


