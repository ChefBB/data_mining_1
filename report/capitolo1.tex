\chapter{Data Understanding and Preparation}
\label{ch:capitolo1}

% --- Inizio del Capitolo 1 ---

% Capitolo 1...

% Esempio di una nota\footnote{CleanCode} con citazione a sezione~\ref{sec:sezione1}.

% \begin{figure}[H]
%     \centering
%     \includegraphics[width=0.2\textwidth]{immagini/logo-dip_blu_hr.png}
%     \caption{Esempio di un'immagine}
%     \label{fig:immagine1}
% \end{figure}

% \begin{lstlisting}[caption={Esempio codice JavaScript},label={Esempio codice JavaScript}, language=JavaScript]
% function foo(num) {
%     const bar = 2;
%     return num + bar;
% }

% const result = foo(2);
% // result -> 4
% \end{lstlisting}


% \section{Sezione 1}\label{sec:sezione1}


% Sezione 1... con citazione bibliografica~\cite{CleanCode}

% \subsection{Sottosezione 1}
% \label{subsec:sottosezione1}

% Sottosezione 1...

\section{Data Introduction}\label{sec:data_intro}
% introduction to each column, how they store the data, etc...
% also should contain some graph to show the distributions etc...

% ['originalTitle', 'rating', 'startYear', 'endYear', 'runtimeMinutes',
% 'awardWins', 'numVotes', 'worstRating', 'bestRating', 'totalImages',
% 'totalVideos', 'totalCredits', 'criticReviewsTotal', 'titleType',
% 'awardNominationsExcludeWins', 'canHaveEpisodes', 'isRatable',
% 'isAdult', 'numRegions', 'userReviewsTotal', 'ratingCount',
% 'countryOfOrigin', 'genres']

% [Bruno: I think we can discuss correlations and eliminations here]
% e.g.:
% <variable1>
% <Paragraph about var1>
% 
% <variable2>, <variable3>
% Paragraph about var2, 3 and why they are correlated and can be eliminated


The dataset \textit{complete\_df.csv}, which is the merge of the original \textit{train.csv} and \textit{test.csv} datasets, contains 21909 titles of different forms of visual entertainment that have been rated on IMDb, an online database of information related to films, television series etc. 
Each record is described by 23 attributes, both numerical and non-numerical. 
All the variables of the dataset are introduced and explained in Table 1.1 and Table 1.2.
\begin{table}[h!]
    \centering
    \begin{tabularx}{\textwidth}{|l|l|X|} % Using 'l' for left alignment of columns
        \hline
        \textbf{Attribute} & \textbf{Type} & \textbf{Description} \\ 
        \hline
        originalTitle & Nominal & Title in its original language \\  
        \hline
        rating & Ordinal & IMDB title rating class \\
        & & Range: from (0,1] to (9,10], converted into an integer interval by approximating to ceiling: [1, 10]\\ 
        \hline
        titleType & Nominal & The type of media product \\ 
        \hline
        canHaveEpisodes & Binary & Whether or not the title can have episodes \\ 
        & & True: can have episodes; False: cannot have episodes \\ 
        \hline
        isRatable & Binary & Whether or not the title can be rated by users \\ 
        & & True: it can be rated; False: cannot be rated \\
        & & The training set, as well as the test set, only contain True values; hence, this attribute will be discarded \\
        \hline
        isAdult & Binary & Whether or not the title is for adults \\ 
        & & 0: non-adult title; 1: adult title \\ 
        \hline
        countryOfOrigin & Nominal & The country(ies) where the title was produced \\ 
        \hline
        genres & Nominal & The genre(s) associated with the title \\ 
        \hline
    \end{tabularx}
    \caption{Description of non-numerical attributes}
    \label{tab:attributes}
\end{table}

% \renewcommand{\arraystretch}{2} % Adjust row height
\begin{table}[h!]
    \centering
    \begin{tabularx}{\textwidth}{|l|l|X|} % Using 'l' for left alignment of columns
        \hline
        \textbf{Attribute} & \textbf{Type} & \textbf{Description} \\ 
        \hline
        runtimeMinutes & Integer & Runtime of the title expressed in minutes \\ 
        \hline
        startYear & Year & Release/start year of a title \\ 
        \hline
        endYear & Year & TV Series' end year \\
        \hline
        awardWins & Integer & Number of awards the title won \\ 
        \hline
        numVotes & Integer & Number of votes the title has received \\ 
        \hline
        worstRating & Integer & Worst title rating \\
        & & Range: [1, 10] \\
        & & Always equal to 1, so the column was discarded \\
        \hline
        bestRating & Integer & Best title rating \\ 
        & & Range: [1, 10] \\
        & & Always equal to 10, so the column was discarded \\
        \hline
        totalImages & Integer & Number of Images on the IMDb title page \\ 
        \hline
        totalVideos & Integer & Number of Videos on the IMDb title page \\ 
        \hline
        totalCredits & Integer & Number of Credits for the title \\ 
        \hline
        criticReviewsTotal & Integer & Total Number of Critic Reviews \\ 
        \hline
        awardNominationsExcludeWins & Integer & Number of award nominations excluding wins \\ 
        \hline
        numRegions & Integer & The number of regions where this version of the title is available \\ 
        \hline
        userReviewsTotal & Integer & Number of User Reviews \\ 
        \hline
        ratingCount & Integer & The total number of user ratings for the title \\
        \hline
    \end{tabularx}
    \caption{Description of numerical attributes}
    \label{tab:numerical_attributes}
\end{table}
% \renewcommand{\arraystretch}{1.5} % Adjust row height


\section{Distribution of the variables and statistics}\label{sec:variable_distrib}
\subsection{Discrete attributes}
...content...
...content...
...content...
...content...
...content...
...content...
...content...
...content...
...content...
...content...

\subsection{Continuous attributes}
...content...
...content...
...content...
...content...
...content...
...content...
...content...
...content...
...content...
...content...