\chapter{Pattern Mining}
\label{ch:capitolo2}
The pattern mining technique chosen for this task was Apriori.
To perform this task, continuous attributes were discretized according to
their distributions. The objective of this process was creating bins that were both meaningful
and balanced in terms of number of attributes for each bin.
Among all the available attributes, the ones chosen for the pattern mining task are the following (with their corresponding binning):
\textbf{AGGIUSTARE QUANDO VERRANNO GESTITI OUTLIER e quando avremo var definitive - cambiare nome var runtimebruno}
\begin{itemize}
    \item \texttt{runTimeMinutes\_Bruno}: VeryLowRT (0-25), LowRT (26-60), MediumRT (61-120), HighRT (121-180), VeryHighRT (181-3000)
    \item \texttt{numVotes}: VeryLowV (5-15), LowV (16-50), MediumV (51-150), HighV (151-997), VeryHighV (1001-966565)
    \item \texttt{rating}: VeryLowR (1-3), LowR (4-6), MediumR (7), HighR (8), VeryHighR (9-10)
    \item \texttt{userReviewsTotal}: NoUR (0), FewUR (1-2), ManyUR (4-13), VeryManyUR (31-149)
    \item \texttt{countryOfOrigin\_EU, \_NA, \_OC, \_AS, \_AF, \_SA}: not\_from\_[continent name], is\_from\_[continent name]
\end{itemize}
In addition, \texttt{titleType} was \textbf{come l'abbiamo gestita per PM? non c'è one hot encoding quindi l'abbiamo usata così com'è}.
The data on which this task was performed was not normalized.


\section{Extraction of frequent patterns}\label{sec:freq_patterns}


\section{Extraction of rules}\label{sec:rules}



\section{Exploiting rules for target prediction}\label{sec:prediction_rules}
