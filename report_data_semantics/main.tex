% Carattere dimensione 12
\documentclass[10pt,openany]{report}

\setlength{\parindent}{0pt}

% Per la stampa fronte-retro sostituire con:
% \documentclass[12pt, twoside]{report}

% Margini (4cm a sx, 2.5cm a dx, 2.5cm in alto, 2.5cm in basso)
\usepackage[top=2cm, bottom=2cm, left=2cm, right=2cm, centering]{geometry}
\usepackage{wrapfig}


% Per la stampa fronte-retro sostituire con: 
% \usepackage[top=2.5cm, bottom=2.5cm, inner=4cm, outer=4cm, right=2.5cm, centering]{geometry}

% Interlinea
\linespread{1.2}

\usepackage{booktabs}

% Librerie utili
\usepackage[english]{babel} % applicazione regole di scrittura per la lingua italiana 
\usepackage[utf8]{inputenc} % codifica UTF-8
\usepackage{scrlayer-scrpage} % stili pagina per il frontespizio
\ifoot[]{}
\cfoot[]{}
\cfoot[\pagemark]{\pagemark}
\pagestyle{scrplain}
\usepackage{lmodern} % font Times New Roman (simile)
\usepackage{graphicx} % inserimento di immagini
\usepackage{subcaption}
\usepackage{csquotes} % per le citazioni "in blocco"
\usepackage[backend=biber, sorting=none, ]{biblatex} % bibliografia con pacchetto biblatex (https://ctan.org/pkg/biblatex?lang=en)
\addbibresource{bibliography.bib}
\appto{\bibsetup}{\raggedright}


\usepackage{titlesec} % per la formattazione dei titoli delle sezioni, capitoli etc.
\usepackage{float} % per il posizionamento delle immagini

\usepackage{listings} % per il codice di programmazione
% Fonte https://en.wikibooks.org/wiki/LaTeX/Source_Code_Listings. Per la lista di sintassi riconosciute.
\renewcommand{\lstlistingname}{Code}% Listing -> Codice
\usepackage{xcolor}  % stile del codice
\definecolor{mygreen}{rgb}{0,0.6,0}
\definecolor{mygray}{rgb}{0.5,0.5,0.5}
\definecolor{mymauve}{rgb}{0.58,0,0.82}
\definecolor{darkgray}{rgb}{.4,.4,.4}
\definecolor{navy}{HTML}{000080}
\definecolor{purple}{rgb}{0.65, 0.12, 0.82}
\definecolor{codepurple}{rgb}{0.58,0,0.82}
\definecolor{backcolour}{rgb}{0.95,0.95,0.92}

% \usepackage{longtable}
\usepackage{tabularx}
\usepackage{ragged2e}
\usepackage{multicol} % Per supportare colonne multiple




% Stili configurabili del codice (lslisting) 
\lstset{ %
belowcaptionskip=0.5em,
backgroundcolor=\color{backcolour}, % choose the background color; you must add \usepackage{color} or \usepackage{xcolor}
basicstyle=\footnotesize, % the size of the fonts that are used for the code
breakatwhitespace=false, % sets if automatic breaks should only happen at whitespace
breaklines=true, % sets automatic line breaking
captionpos=b, % sets the caption-position to bottom
commentstyle=\color{mygreen}, % comment style
deletekeywords={...}, % if you want to delete keywords from the given language
escapeinside={\%*}{*)}, % if you want to add LaTeX within your code
extendedchars=true, % lets you use non-ASCII characters; for 8-bits encodings only, does not work with UTF-8
frame=single, % adds a frame around the code
keepspaces=true, % keeps spaces in text, useful for keeping indentation of code (possibly needs columns=flexible)
keywordstyle=\color{codepurple}, % keyword style
% language=Octave, % the language of the code
morekeywords={*,...}, % if you want to add more keywords to the set
numbers=left, % where to put the line-numbers; possible values are (none, left, right)
numbersep=5pt, % how far the line-numbers are from the code
numberstyle=\tiny\color{mygray}, % the style that is used for the line-numbers
rulecolor=\color{black}, % if not set, the frame-color may be changed on line-breaks within not-black text (e.g. comments (green here))
showspaces=false, % show spaces everywhere adding particular underscores; it overrides 'showstringspaces'
showstringspaces=false, % underline spaces within strings only
showtabs=false, % show tabs within strings adding particular underscores
stepnumber=1, % the step between two line-numbers. If it's 1, each line will be numbered
stringstyle=\color{mymauve}, % string literal style
tabsize=2, % sets default tabsize to 2 spaces
title=\lstname % show the filename of files included with \lstinputlisting; also try caption instead of title
}


% Formato delle intestazioni
% \titleformat{\chapter}[hang]{\normalfont\Large\bfseries}{\thechapter.}{0.5em}{}


\makeatletter
\renewcommand{\chapter}{\@startsection{chapter}{0}{0pt}
  {-7ex plus -1ex minus -.2ex}
  {2ex plus .2ex}
  {\normalfont\Large\bfseries}}
\makeatother

% \titlespacing*{\chapter}{0pt}{3em}{1.5em}


\titleformat{\chapter}[hang]          % no "Chapter", just number and title
  {\normalfont\Large\bfseries}       % formatting of the whole line
  {\thechapter.}                     % numbered like "1."
  {0.5em}                            % spacing between number and title
  {}                                 % what to put before the title text

\titlespacing*{\chapter}             % controls vertical spacing
  {0pt}                              % left margin
  {3em}                              % space *before* chapter title
  {1.5em} 


% % Prevent chapters from starting on a new page
% \makeatletter
% \renewcommand{\@makechapterhead}[1]{%
%   {\parindent \z@ \raggedright \normalfont
%   \interlinepenalty\@M
%   \Huge\bfseries \thechapter. #1\par\nobreak
%   \vskip 20pt }}
% \makeatother

% % Optional: If you also want unnumbered chapters to not start on a new page:
% \makeatletter
% \renewcommand{\@makeschapterhead}[1]{%
%   {\parindent \z@ \raggedright
%   \normalfont
%   \Huge\bfseries #1\par\nobreak
%   \vskip 20pt }}
% \makeatother

% % Chapter format: no "Chapter" word, just "1. Title"
% \titleformat{\chapter}[hang]
%   {\normalfont\Large\bfseries}
%   {\thechapter.}
%   {0.5em}
%   {}

% % Spacing: left indent, before title, after title
% \titlespacing*{\chapter}
%   {0pt}
%   {5em}  % space *before* title
%   {2em}  % space *after* title


\begin{document}


% Frontespizio
% \begin{titlepage}
% \begin{figure}
%     \centering\includegraphics[scale=0.5]{immagini/cherubino_pant541.png}
% \end{figure}
\pagestyle{empty}         % niente numero di pagina
\pagenumbering{gobble}    % disattiva numerazione
\begin{center}  
     {\LARGE { Data Mining: Fundamentals}}\\
     \vspace{2cm}
    {\Large { A.Y. 2024/2025 }}\\
    \vspace{2cm}
    {\Large { Group 12 }}\\
     \vspace{2cm}
     {\large { Bruno Barbieri, Noemi Dalmasso, Gaia Federica Francesca Ferrara }}
\end{center}

% The aim of this report is to display an analysis carried out on the IMDb dataset; the analysis has been conducted making use of data mining methodologies. 
% After the data understanding and preparation phase, clustering, classification, regression and pattern mining techniques have been applied.

% \vspace{2cm}

% \begin{minipage}[t]{0.47\textwidth}
%   {\large{\bf Relatore:\\ Nome Cognome}}
% 	\vspace{0.5cm}
% 	{\large{\bf \\Correlatore:\\ Nome Cognome}}
% \end{minipage}\hfill\begin{minipage}[t]{0.47\textwidth}\raggedleft
%   {\large{\bf Group 12\\ }}
%   \vspace{0.5cm}
  
% \end{minipage}

% \vspace{25mm}

% \centering{\large{\bf ANNO ACCADEMICO 20xx/20xx }}
% \end{titlepage}
% Fine frontespizio
\clearpage

\tableofcontents
\clearpage

\pagenumbering{arabic}  % numerazione araba
\setcounter{page}{1}    % riparti da pagina 1
\pagestyle{scrplain}    % stile pagina con numeri al centr

% \addtocontents{toc}{\protect\thispagestyle{empty}}
% \addcontentsline{toc}{chapter}{Introduction} % Capitolo non numerato
%\chapter*{Introduction}
\label{ch:introduzione}



\chapter{Data Understanding and Preparation}
\label{ch:capitolo1}

The dataset \textit{train.csv} contains 16431 titles of different forms of visual entertainment that have been rated on IMDb, 
an online database of information related to films, television series etc. 
Each record is described by 23 attributes, either discrete or continuous. 
% All the variables of the dataset are introduced and explained in Table 1.1 and Table 1.2.


% \section{Distribution of the variables and statistics}\label{sec:variable_distrib}
% This section will give an overview about the distribution of variables that has been carried on to understand patterns, 
% detect meaningful statistics and assess their relevance to the project. 

\section{Discrete Attributes}
Table~\ref{tab:attributes} shows the discrete attributes of the dataset,
their types and a brief description of each attribute.
\begin{table}[h]
\centering
\begin{tabular}{lll}
\toprule
\textbf{Attribute} & \textbf{Type} & \textbf{Description} \\
\midrule
\texttt{originalTitle} & Categorical & Title in its original language \\
\texttt{rating} & Ordinal & IMDB title rating class, from \texttt{(0,1]} to \texttt{(9,10]} \\
\texttt{worstRating} & Ordinal & Worst title rating \\
\texttt{bestRating} & Ordinal & Best title rating \\
\texttt{titleType} & Categorical & The format of the title \\
\texttt{canHaveEpisodes} & Binary & Whether the title can have episodes: \texttt{True}/\texttt{False} \\
\texttt{isRatable} & Binary & Whether the title can be rated: \texttt{True}/\texttt{False} \\
\texttt{isAdult} & Binary & Whether the title is adult content: \texttt{0} (non-adult), \texttt{1} (adult) \\
\texttt{countryOfOrigin} & List & Countries where the title was produced \\
\texttt{genres} & List & Genre(s) associated with the title (up to 3) \\
\bottomrule
\end{tabular}
\caption{Description of discrete attributes}
\label{tab:attributes}
\end{table}

% \begin{table}[h]
%     \centering
%     \begin{tabular}{|l|l|l|} % Using 'l' for left alignment of columns
%         \hline
%         \textbf{Attribute} & \textbf{Type} & \textbf{Description} \\ 
%         \hline
%         \texttt{originalTitle} & Categorical & Title in its original language \\  
%         \hline
%         \texttt{rating} & Ordinal & IMDB title rating class \\
%         & & The range is from \texttt{(0,1]} to \texttt{(9,10]} \\ 
%         \hline
%         \texttt{worstRating} & Ordinal & Worst title rating \\ 
%         \hline
%         \texttt{bestRating} & Ordinal & Best title rating \\ 
%         \hline
%         \texttt{titleType} & Categorical & The format of the title \\ 
%         \hline
%         \texttt{canHaveEpisodes} & Binary & Whether or not the title can have episodes \\ 
%         & & \texttt{True}: can have episodes; \texttt{False}: cannot have episodes \\ 
%         \hline
%         \texttt{isRatable} & Binary & Whether or not the title can be rated by users \\ 
%         & & \texttt{True}: it can be rated; \texttt{False}: cannot be rated \\ 
%         \hline
%         \texttt{isAdult} & Binary & Whether or not the title is for adults \\ 
%         & & \texttt{0}: non-adult title; \texttt{1}: adult title \\ 
%         \hline
%         \texttt{countryOfOrigin} & List & The country(ies) where the title was produced \\ 
%         \hline
%         \texttt{genres} & List & The genre(s) associated with the title (3 at most) \\ 
%         \hline
%     \end{tabular}
%     \caption{Description of discrete attributes}
%     \label{tab:attributes}
% \end{table}
\subsection{Merging and Removal of Discrete Attributes}\label{subsec:var_elim_discrete}
The following discrete attributes were removed from the dataset:
\begin{itemize}
    \item \texttt{originalTitle} was removed because it is not relevant for the analysis;
    \item the \texttt{isRatable} variable was removed because all the titles in the dataset are ratable;
    \item \texttt{worstRating} and \texttt{bestRating} attributes were removed because they assume the same values for all records (1 and 10 respectively).
\end{itemize}

Additionally, the \texttt{isAdult} attribute is highly correlated with the presence or absence of
\textit{Adult} in \texttt{genre} (16 records differ in the train set, 1 in the test set), so the two were
merged with a logical OR operation. This is not true for the \textit{short} type in \texttt{titleType}, with
491 records having different values from the obtained feature. For this reason, the two were kept separate.



\subsection{Discrete Attributes Analysis}
This paragraph provides an overview of the discrete attributes in the dataset, focusing on their distributions and statistics.
The following figures~\ref{fig:titleType_distrib} and~\ref{fig:rating_distrib} show bar plots of \texttt{titleType} and \texttt{rating} attributes, respectively.\\
From figure~\ref{fig:titleType_distrib} it is observed that the classes of the titleType attribute are unbalanced, with \textit{movie} being the most frequent class (5535 records).
It was observed that the class \textit{tvShort} is the least frequent in the dataset, with only 40 records (around 0.24\% of the dataset). Because of this, these rows were discarded from the dataset, as they were considered irrelevant for the analysis.
The decision was not repeated for \textit{tvSpecial} and \textit{tvMiniSeries}, as they cover slightly more than 1\% of the dataset each (166, 1.01\% and 224, 1.36\%, respectively). \\
\begin{figure}[H]
    \centering
    % First subfigure
    \begin{subfigure}{0.49\textwidth}
        \centering
        \includegraphics[width=0.98\textwidth]{plots/types_count.png}     %se teniamo 0.65 ci sta sotto la tabella delle continuous
        \caption{Distribution of \texttt{titleType}}
        \captionsetup{width=0.9\linewidth, justification=centering}
        \label{fig:titleType_distrib}
    \end{subfigure}
    \begin{subfigure}{0.49\textwidth}
        \centering
        \includegraphics[width=0.98\textwidth]{plots/rating_distrib.png}     %se teniamo 0.65 ci sta sotto la tabella delle continuous
        \caption{Distribution of \texttt{rating}}
        \captionsetup{width=0.9\linewidth, justification=centering}
        \label{fig:rating_distrib}
    \end{subfigure}
    \captionsetup{justification=centering}
    \caption{Distribution of the \texttt{titleType} and \texttt{rating} attributes}
    \label{fig:distrib}
\end{figure}

As shown in figure~\ref{fig:rating_distrib}, the \texttt{rating} attribute roughly follows a normal distribution, with a slightly asymmetric peak:
a significant number of titles falls within the (6,7] and (7, 8] ranges (4565 and 4822 titles, respectively) while only a total amount of 67 titles falls within (0,1] and (1,2].
% The overall distribution resembles a normal distribution, with a slightly asymmetric peak.


\subsection{Encoding and Transformation of Categorical Attributes}
The attribute \texttt{rating} was transformed by taking the upper bound of each rating
interval's string representation. This approach was chosen because the minimum rating is 1, meaning the
lowest interval corresponds only to ratings of 1. For consistency, the same transformation was applied
to all other intervals.\\

Multi-label one-hot encoding was applied to the \texttt{genres} column. 
Each unique genre was represented as a binary feature, allowing records that belong to multiple genres simultaneously to maintain this information; this generated 28 new features.
Depending on the task, some were often discarded to avoid overfitting or to reduce the number of features.
This will be discussed in the corresponding sections.
Rows with no genres were assigned a vector of all zeros, indicating the absence of any genres.\\

% After that, multi-label one-hot encoding was applied to the
% \texttt{genres} column; each unique genre was represented as a binary feature, 
% allowing records that belong to multiple genres simultaneously to maintain this information.
% \vspace{1em}
The attribute \texttt{countryOfOrigin} was represented by grouping the countries by continent.
The following variables have been created: 
\begin{multicols}{2}
    \begin{itemize}
        \item \texttt{countryOfOrigin\_AF} (Africa);
        \item \texttt{countryOfOrigin\_AS} (Asia);
        \item \texttt{countryOfOrigin\_EU} (Europe);
        \item \texttt{countryOfOrigin\_NA} (North America);
        \item \texttt{countryOfOrigin\_SA} (South America);
        \item \texttt{countryOfOrigin\_OC} (Oceania);
        \item \texttt{countryOfOrigin\_UNK} (Unknown country);
        \item \texttt{countryOfOrigin\_freq\_enc} (frequency encoding of the original list).
    \end{itemize}
\end{multicols}


For each record, the first six features provide the number of countries for each continent.\\
The \texttt{countryOfOrigin\_UNK} variable counts the number of countries that are not recognized as belonging to a continent for that record.\\

% Each of the first six features provides the number of countries in the corresponding continent.\\
% \texttt{countryOfOrigin\_UNK} is used to represent the strings that are not categorized as being part of a
% continent, by counting the strings that are not recognized.
Additionally, \texttt{countryOfOrigin\_freq\_enc} provides the frequency encoding of the original list of countries as a whole, 
showing how frequently a specific combination of countries appears across the entire dataset.
% In summary, the original attribute is represented by the seven features regarding
% the continents, plus 1 representing the frequency encoding.
These transformations allow to keep a most of the original information, while limiting the number of new features.




\section{Continuous Attributes}
Table~\ref{tab:numerical_attributes} shows the continuous attributes of the dataset, their type and
a brief description.
\vspace{1em}
\begin{table}[h]
\centering
\begin{tabular}{lll}
\toprule
\textbf{Attribute} & \textbf{Type} & \textbf{Description} \\
\midrule
\texttt{runtimeMinutes} & Integer & Runtime of the title expressed in minutes \\
\texttt{startYear} & Integer & Release/start year of a title \\
\texttt{endYear} & Integer & TV Series end year \\
\texttt{awardWins} & Integer & Number of awards the title won \\
\texttt{numVotes} & Integer & Number of votes the title has received \\
\texttt{totalImages} & Integer & Number of images on the IMDb title page \\
\texttt{totalVideos} & Integer & Number of videos on the IMDb title page \\
\texttt{totalCredits} & Integer & Number of credits for the title \\
\texttt{criticReviewsTotal} & Integer & Total number of critic reviews \\
\texttt{awardNominationsExcludeWins} & Integer & Number of award nominations excluding wins \\
\texttt{numRegions} & Integer & Number of regions for this version of the title \\
\texttt{userReviewsTotal} & Integer & Number of user reviews \\
\texttt{ratingCount} & Integer & Total number of user ratings for the title \\
\bottomrule
\end{tabular}
\caption{Description of continuous attributes}
\label{tab:numerical_attributes}
\end{table}

% \begin{table}[h]
%     \centering             
    
    %nel type secondo me ha senso mettere il tipo di variabile continua (quindi interval, ratio ecc.)        

%     \begin{tabular}{|l|l|l|} 
%         \hline
%         \textbf{Attribute} & \textbf{Type} & \textbf{Description} \\
%         \hline
%         \texttt{runtimeMinutes} & Integer & Runtime of the title expressed in minutes \\ 
%         \hline
%         \texttt{startYear} & Integer & Release/start year of a title \\ 
%         \hline
%         \texttt{endYear} & Integer & TV Series end year \\
%         \hline
%         \texttt{awardWins} & Integer & Number of awards the title won \\ 
%         \hline
%         \texttt{numVotes} & Integer & Number of votes the title has received \\ 
%         \hline
%         \texttt{totalImages} & Integer & Number of Images on the IMDb title page \\ 
%         \hline
%         \texttt{totalVideos} & Integer & Number of Videos on the IMDb title page \\ 
%         \hline
%         \texttt{totalCredits} & Integer & Number of Credits for the title \\ 
%         \hline
%         \texttt{criticReviewsTotal} & Integer & Total Number of Critic Reviews \\ 
%         \hline
%         \texttt{awardNominationsExcludeWins} & Integer & Number of award nominations excluding wins \\ 
%         \hline
%         \texttt{numRegions} & Integer & The regions number for this version of the title \\ 
%         \hline
%         \texttt{userReviewsTotal} & Integer & Number of User Reviews \\ 
%         \hline
%         \texttt{ratingCount} & Integer & The total number of user ratings for the title \\ 
%         \hline
%     \end{tabular}
%     \caption{Description of continuous attributes}
%     \label{tab:numerical_attributes}
% \end{table}

\subsection{Removal and Merging of Continuous Attributes}\label{sec:var_elim_creation}
% The plot in figure~\ref{fig:correlation_matrix} is a Pearson's correlation matrix that takes into
% account the continuous attributes of the dataset.\\

% \begin{figure}[H]
%     \centering
%     \includegraphics[width=0.65\textwidth]{plots/correlation_matrix.png}
%     \caption{Correlation matrix}
%     \label{fig:correlation_matrix}
% \end{figure}

% The correlation matrix of the continuous features shows that \texttt{ratingCount} and \texttt{numVotes} are perfectly correlated;
% for their redundancy, \texttt{ratingCount} was discarded.\\

The plot in figure~\ref{fig:correlation_matrix} is a Pearson's correlation matrix that takes into
account the continuous attributes of the dataset.
The matrix shows that \texttt{ratingCount} and \texttt{numVotes} are perfectly correlated;
for their redundancy, \texttt{ratingCount} was discarded.\\

The attributes \texttt{awardNominationsExcludeWins} and \texttt{awardWins} were combined into 
\texttt{totalNominations}, due to their strong semantic similarity and high correlation (0.69).
    % correlazione da matrix è 69
The new feature represents the sum of the two original attributes. This transformation also helps
mitigate the impact of their heavy right skew (shown in figure \ref{fig:sub1}), resulting in a more meaningful and interpretable feature.\\
% \end{minipage}
% \hfill

% \vspace{2em}
Similarly, the \texttt{totalVideos} and \texttt{totalImages} attributes were combined into a single
feature, i.e. \texttt{totalMedia}, representing the total number of media items associated with a title.
Although the original attributes are not highly correlated, both exhibit skewed distributions (as in figure \ref{fig:sub2}), \texttt{totalVideos} in particular.
Due to this, and to their similar semantic meaning, they were merged to form a more consolidated and
interpretable feature.
% \vspace{1em}


% codice due colonne
% \noindent
% \begin{minipage}{0.50\textwidth}
\begin{figure}[H]
    \centering
    \includegraphics[width=0.70\textwidth]{plots/correlation_matrix.png}
    \captionof{figure}{Correlation matrix}
    \label{fig:correlation_matrix}
\end{figure}
% \end{minipage}
% \hfill
% \begin{minipage}{0.47\textwidth}


%     The attributes \texttt{awardNominationsExcludeWins} and \texttt{awardWins} were combined into a single
%     feature, i.e. \texttt{totalNominations}, due to their strong semantic similarity and high correlation (0.95).
%     % correlazione da matrix è 69
%     The new feature represents the sum of the two original attributes. This transformation also helps
%     mitigate the impact of their heavy left skew (shown in figure \ref{fig:sub1}), resulting in a more meaningful and interpretable feature.\\\\
% Similarly, the \texttt{totalVideos} and \texttt{totalImages} attributes were combined into a single
% feature, i.e. \texttt{totalMedia}, representing the total number of media items associated with a title.

% Although the original attributes are not highly correlated, both exhibit skewed distributions (as shown in figure \ref{fig:sub2}), \texttt{totalVideos} in particular.
% Due to this, and their similar semantic meaning, they were merged to form a more consolidated and
% interpretable feature.\\




\begin{figure}[H]
    \centering
    % First subfigure
    \begin{subfigure}{0.43\textwidth}
        \includegraphics[width=\textwidth]{plots/nominations_distrib.png}
        \captionsetup{width=0.9\linewidth, justification=centering}
        \caption{Kernel Density Estimation of \texttt{awardWins} and \texttt{awardNominationsExcludeWins}}
        \label{fig:sub1}
    \end{subfigure}
    \begin{subfigure}{0.43\textwidth}
        \includegraphics[width=\textwidth]{plots/totalVideos_Images_distrib.png}
        \captionsetup{width=0.9\linewidth, justification=centering}
        \caption{Kernel Density Estimation of \texttt{totalVideos} and \texttt{totalImages}}
        \label{fig:sub2}
    \end{subfigure}
    \captionsetup{justification=centering}
    \caption{Distribution of the attributes that form the \texttt{totalNominations} and \texttt{totalMedia} features}
    \label{fig:distrib}
\end{figure}

Although \texttt{criticReviewsTotal} and \texttt{userReviewsTotal} also have a relatively high correlation (0.65), as well as a right-skewed distribution, it was decided that the two attributes should be kept separate because of their relevance in meaning. It is also worth noting that the two have high correlations with \texttt{numVotes} (0.67 and 0.75 respectively), but they were all kept because of the difference between votes and reviews.



\section{Data Quality}\label{sec:data_quality}
Next, a proper evaluation of the observed data was conducted in preparation for the analysis.
Once having checked that there are no duplicates and no incomplete rows in the dataset,
attention was given at identifying missing values and outliers.

% parte che ci sembrava importante per far vedere che ce ne siamo accorti
% \subsection{Syntactic Inconsistencies} 
% Even though \texttt{awardWins} was the only feature having missing values marked with \texttt{NaN}, 
% it has been noticed that there were missing values also in other columns - \texttt{endYear}, \texttt{runtimeMinutes} and \texttt{genres} -
% marked with the string "\textbackslash N" instead.
% To avoid this inconsistency those values have been replaced with \texttt{NaN}.


\subsection{Missing Values}\label{sec:missing_values}
% The missing values in the above-mentioned attributes were handled as follows:
The following attributes were found to have missing values\footnote{\texttt{awardWins} was the only feature with missing values marked as \texttt{NaN}, while the other listed columns had missing values marked as "\textbackslash N", hence they were replaced with \texttt{NaN}.}
:
\begin{itemize}
    \item \texttt{endYear}: it is the feature with the highest number of \texttt{NaN} values (15617; about 95\%).
    Although the feature is only relevant for \textit{TVSeries} and \textit{TVMiniSeries} titles, it still
    had approximately 50\% missing values within those categories, limiting its usefulness even in the
    appropriate context. For this reason, the feature was discarded.
    
    \item \texttt{runtimeMinutes}: this attribute has 4,852 missing values (29.5\%). Two imputation strategies were employed, both based on random sampling within the interquartile range. 
    One strategy used the \texttt{titleType} feature to define the range, while the other imputed values based off of the attribute's distribution alone. 
    The choice of which of the two strategies to use depends on the specific task, and will be specified in the corresponding sections.
    
    \item \texttt{awardWins}: this feature has 2618 \texttt{NaN} values (about 16\%).
    Since the mode associated with this variable is 0, it has been decided to substitute the missing
    values with 0.

    \item \texttt{genres}: it has 382 missing values (2.3\%). Having dealt this variable with a
    multi-label one-hot encoding process (as has been described in the \textit{Encoding and Transformation of categorical attributes}
    section), a vector of all zeros is assigned to record with missing genres values.
\end{itemize}



\subsection{Semantic Inconsistencies, Feature Transformations and Outlier detection}
While analyzing the dataset, it was observed that the \textit{Videogame} type of the \texttt{titleType} attribute (259 records - around 1.58\% of the dataset) 
was not consistent with the other values of the same feature, being \textit{Videogame} a fundamentally different titleType.
Other then this semantic inconsistency, these rows generated problems for some of the other attributes, such as \texttt{runtimeMinutes}, resulting in most values being missing and difficult to impute. 
Because of this, the samples were removed from the dataset. \\


Some features showed a heavy right-skewed distribution, with typical traits of Power-Law Distributions. Their Kernel Density Estimations are shown in figure~\ref{fig:left_skewed}.
\vspace{1em}
\begin{figure}[H]
    \centering
    \begin{subfigure}{0.48\textwidth}
        \includegraphics[width=\textwidth]{plots/left_skew_distribs.png}
        \captionsetup{width=0.9\linewidth, justification=centering}
        \caption{KDE of right-skewed attributes}
        \label{fig:sub1_KDE_left_skew}
    \end{subfigure}
    \begin{subfigure}{0.48\textwidth}
        \includegraphics[width=\textwidth]{plots/left_skew_distribs_log.png}
        \captionsetup{width=0.9\linewidth, justification=centering}
        \caption{KDE of right-skewed attributes with log density}
        \label{fig:sub2_KDE_left_skew}
    \end{subfigure}
    \caption{Kernel Density Estimation of the left-skewed attributes}
    \label{fig:left_skewed}
\end{figure}

The decay of these features is exponential in linear space (~\ref{fig:sub1_KDE_left_skew}), while in logarithmic space there is a decline that can be approximated to a linear trend (~\ref{fig:sub2_KDE_left_skew}). 
For this reason, when needed, a log-transformation was applied to these attributes to reduce the skewness and make
them more suitable for some specific analysis.
Because of right-skewness (without a power-law distribution), other attributes were also log-transformed:
\begin{itemize}
    \item \texttt{numVotes};
    \item \texttt{totalCredits};
    \item \texttt{totalMedia}.
\end{itemize}


Regarding outliers, the feature that was found to be more problematic was \texttt{runtimeMinutes}.
Similarly to missing values imputation (~\ref{sec:missing_values}),
outlier detection was performed using two different strategies:
% Figure~\ref{fig:runtimeMinutes_boxplot} shows the first approach, which computes outliers on each \texttt{titleType} separately.
the first approach computes outliers on each \texttt{titleType} separately,
while the second is based on the distribution of the \texttt{runtimeMinutes} attribute alone.
Figure~\ref{fig:outliers} reports an analysis of the feature through the IQR method separately on each type.
% Figure ~\ref{fig:runtimeMinutes_boxplot_no_type} shows the second, which computes them based on \texttt{canHaveEpisodes} attribute and \textit{Short} genre.

\begin{figure}[H]
    \centering
    % \begin{subfigure}{0.58\textwidth}
        % \includegraphics[width=\textwidth]{plots/outliers.png}
        \includegraphics[width=0.6\textwidth]{plots/outliers.png}
        % \captionsetup{width=0.9\linewidth, justification=centering}
        \caption{Boxplot of the \texttt{runtimeMinutes} attribute for each \texttt{titleType}}
        \label{fig:runtimeMinutes_boxplot}
    % \end{subfigure}
    % \begin{subfigure}{0.4\textwidth}
    %     \includegraphics[width=\textwidth]{plots/outliers_notype.png}
    %     \captionsetup{width=0.9\linewidth, justification=centering}
    %     \caption{Boxplot of the \texttt{runtimeMinutes} attribute without \texttt{titleType}}
    %     \label{fig:runtimeMinutes_boxplot_no_type}
    % \end{subfigure}
    \captionsetup{justification=centering}
    % \caption{Outliers in the \texttt{runtimeMinutes} attribute}
    \label{fig:outliers}
\end{figure}

The boxplots show that there are samples that have been misreported, with runtimes of over 1000 minutes for \textit{tvSeries}.
% the same can be observed on the second plot on the \textit{Has episodes} box plot.
This might be because of an inconsistency with the understanding of the meaning of the attribute, and in those cases it might be possible that
the value refers to the total runtime of the series, rather than the runtime of a single episode.\\


% Another interesting observation regards the presence of records with a runtime of 0 minutes for the \textit{short} type. This was present in just 1 record, and it was removed from the dataset because it was considered a mistake, although it was not considered an outlier.\\
Another interesting observation regards the presence of a record with a runtime of 0 minutes for the \textit{short} type;
% although it was not considered an outlier by the first method,
the record was removed from the dataset because it was regarded as an erroneous sample.\\

In this phase, outliers were not removed from the dataset by default. Instead, a case-by-case approach was adopted,
testing each task and analysis both with and without the outliers. Notably, in every case, better results were obtained when outliers were excluded. %data preparation

\chapter{Clustering}
\label{ch:capitolo2}

This chapter of the report aims at illustrating the clustering analysis performed on the dataset at hand.
The employed clustering techniques are K-means (Centroid-based), DBSCAN (density-based) and hierarchical clustering.

The analysis conducted using these methods focused only exclusively on the dataset's numerical attributes, which were appropriately log-transformed (as mentioned in the \textit{Variable Transformation} section) and normalized using \texttt{MinMaxScaler}.  
For the K-means algorithm, \texttt{totalNominations} and \texttt{totalMedia} were excluded due to their high proportion of zero values, which negatively affected cluster formation.

In addition, an attempt was made to incorporate categorical variables to the analysis with the K-means algorithm by converting them into binary attributes and constructing a mixed-distances matrix. 
Distances were then calculated using the Euclidean distance for numerical (log-tranformed and scaled) features and the Jaccard similarity for binary ones. 
However, this approach was computationally expensive and did not lead to any improvement in the results.

% \textbf{RIVEDERE PARTE PCA}
% Principal Component Analysis (PCA) was applied to the preprocessed data just for clusters visualization purposes. 
% Analysis of the numerical attributes reveals that 4 principal components are optimal when excluding variables with many zero values, while 5 components are needed when including all variables. 
% These numbers of components capture the maximum meaningful variance, as shown by the point in the plots where where the line starts to flatten, indicating that adding more components doesn't increase explained variance significantly. 
% The plots in figure~\ref{fig:pca_diff} show the differences between these two approaches. \textbf{IN REALTA' NON SI VEDE TROPPO LA DIFFERENZA, QUINDI MAGARI NON FARE PCA MA SOLO VISUALIZZAZIONE CON ISTOGRAMMI??? oppure semplicemente tenere solo uno dei due plot?}
% \begin{figure}[H]
%     \centering
%     \begin{subfigure}[b]{0.40\textwidth}
%         \centering
%         \includegraphics[width=\textwidth]{plots/pca_kmeans.png}
%         \caption{PCA excluding variables}
%         \label{fig:sse_silh_kmeans}
%     \end{subfigure}
%     \begin{subfigure}[b]{0.40\textwidth}
%         \centering
%         \includegraphics[width=\textwidth]{plots/pca5.png}
%         \caption{PCA with all variables}
%         \label{fig:pca5}
%     \end{subfigure}
%     \caption{Principal Component Analysis}
%     \label{fig:pca_diff}
% \end{figure}

\section{K-means}\label{sec:centroid_based}
%The clustering analysis performed with the K-means algorithm focused on the numeric variables of the dataset, excluding \texttt{awardWins}, \texttt{awardNominationsExcludeWins}, and \texttt{totalCredits} due to their high proportion of zero values, which negatively affected cluster formation. 
%The variables have been appropriately log-transformed (as illustrated in the \textit{Variable Transformation} section) and normalized with \texttt{StandardScaler}.

To identify the optimal number of clusters, both the SSE and Silhouette scores were computed. The goal was to find a configuration that minimizes the SSE while maintaining a robust Silhouette score and a proper \textit{k}. 
The plots in figure~\ref{fig:sse_silh_kmeans} demonstrate that \textit{k} = ??? provides the optimal balance between these metrics. Choosing \textit{k} = ??? returns a SSE score of ??? and Silhouette score of ???. 
\textbf{VALORI TROPPO ALTI, VEDERE SE CAMBIANO CAMBIANDO SCALING E/O TOGLIENDO OUTLIER}

% To visualize the clustering results, Principal Component Analysis (PCA) was employed. The plot in figure~\ref{fig:pca_kmeans} reveals that 4 principal components are enough to capture the optimal amount of variance for the selected variables, as evidenced by the the point where the line starts to flatten, indicating that adding more components doesn't increase explained variance significantly.
The cluster results are presented in figure~\ref{fig:pairplot_kmeans}. 

\begin{figure}[H]
    \centering
    \begin{subfigure}[b]{0.49\textwidth}
        \centering
        \includegraphics[width=\textwidth]{plots/sse_silh_kmeans.png}
        \caption{SSE and Silhouette scores}
        \label{fig:sse_silh_kmeans}
    \end{subfigure}
    % \hfill
    % \begin{subfigure}[b]{0.3\textwidth}
    %     \centering
    %     \includegraphics[width=\textwidth]{plots/pca_kmeans.png}
    %     \caption{PCA Analysis}
    %     \label{fig:pca_kmeans}
    % \end{subfigure}
    % \hfill
    \begin{subfigure}[b]{0.49\textwidth}
        \centering
        \includegraphics[width=\textwidth]{plots/pairplot_kmeans.png}
        \caption{K-means Visualization}
        \label{fig:pairplot_kmeans}
    \end{subfigure}
    \caption{K-means clustering analysis}
    \label{fig:three_subplots}
\end{figure}
The distribution of data points across the four clusters is as follows (shown in percentage of data points per cluster):
% Red (0): 51.68\%, Blue (1): 7.42\%, Green (2): 24.22\%, Orange (3): 16.68\%.
% The clusters are not as well-separated in most Principal Component combinations as they are with PC1. 
% In fact, in the other combinations, the clusters tend to overlap and their boundaries are not always clearly distinct.
% This might be an indication that the true clusters have irregular shapes or different densities, resulting in boundaries between them being not clearly defined.



\section{DBSCAN}\label{sec:density_based}
To determine the optimal DBSCAN parameters, the \textit{k\textsuperscript{th} nearest neighbors} method was used: this allows to identify \textit{eps} (the maximum distance between two points for them to be considered neighbors) given the value of \textit{Minpts} (minimum number of points in a neighborhood for a point to be considered a core point).
Initially, \textit{Minpts} was set to 22, following the rule of setting it above twice the number of dimensions. 
However, due to the dataset's unbalanced nature and the sparsity of high-dimensional data, reducing \textit{Minpts} to 11 allowed the formation of smaller clusters while preventing the risk of detecting only one dominant cluster and classifying many minority groups as noise instead of distinct clusters.
To determine \textit{eps}, the k\textsuperscript{th} nearest neighbors plot with \textit{k} = 11 was analyzed (figure ~\ref{fig:DBSCAN_kth_graph}). While the "knee" point suggested an eps of around 0.1, this value would have resulted in excessive noise and a single dominant cluster. 
To address this, eps was set to 1.564, allowing for meaningful connectivity while preserving the detection of smaller clusters without merging them into a single entity. 
The algorithm identified 4 groups in the dataset, including one representing noise (1,753 points). 
The largest cluster contains 13,198 points, while the smaller clusters consist of 733 and 747 points, 
respectively.The  results are shown in figure~\ref{fig:DBSCAN_provvisoria}

To conclude, by adjusting \textit{eps} and \textit{Minpts} appropriately, the clustering results achieved a Silhouette score of 0.139 \textbf{(SIL CONTANDO OUTLIERS)}, indicating little improved cluster separation and reduced noise, which is considered good enough for an unbalanced, high-dimensional dataset.

\begin{figure}[H]
    \centering
    \begin{subfigure}[b]{0.49\textwidth}
        \centering
        \includegraphics[width=\textwidth]{plots/DBSCAN_kth_graph.png}
        \caption{k\textsuperscript{th} nearest neighbors}
        \label{fig:DBSCAN_kth_graph}
    \end{subfigure}
    % \hfill
    % \begin{subfigure}[b]{0.3\textwidth}
    %     \centering
    %     \includegraphics[width=\textwidth]{plots/pca_kmeans.png}
    %     \caption{PCA Analysis}
    %     \label{fig:pca_kmeans}
    % \end{subfigure}
    % \hfill
    \begin{subfigure}[b]{0.49\textwidth}
        \centering
        \includegraphics[width=\textwidth]{plots/DBSCAN_provvisoria.png}
        \caption{DBSCAN Visualization}
        \label{fig:DBSCAN_provvisoria}
    \end{subfigure}
    \caption{DBSCAN clustering analysis}
    \label{fig:three_subplots}
\end{figure}




\section{Hierarchical clustering}\label{sec:hierarchical}
Hierarchical clustering was performed using all linkages (Ward, Average, Complete, Single), with the Euclidean distance metric.
Figure~\ref{fig:hier_clust_stats} shows the results of the analysis, which includes the Silhouette and SSE scores, as well as the maximum and minimum percentage of points per cluster.
\begin{figure}[H]
    \centering
    % Left: Tall figure (a)
    \begin{subfigure}[t]{0.49\textwidth}
        \centering
        \includegraphics[width=0.95\textwidth]{plots/max_min_pctg.png}
        \subcaption{Max/Min percentage of points per cluster}
        \label{fig:max_min_pctg}
    \end{subfigure}
    \hfill
    \begin{subfigure}[t]{0.49\textwidth}
        \centering
        \includegraphics[width=0.95\textwidth]{plots/sil_sse_hierarchical_clust.png}
        \subcaption{Silhouette and SSE scores}
        \label{fig:sil_sse_hierarchical_clust}
    \end{subfigure}
    \caption{Hierarchical clustering metrics for different numbers of clusters}
    \label{fig:hier_clust_stats}
\end{figure}

From figure~\ref{fig:max_min_pctg}, it can be observed that Single linkage produces a single cluster which contains basically all data points. This makes it unsuitable for this use case.
Average and Complete linkages produce a cluster with a high maximum cluster size (above 90\% of the dataset for all the number of clusters tested for Average, and 80\% for Complete).
These results are likely due to two main causes: 
\begin{itemize}
    \item Usage of skewed features, which lead to areas with a very high density of points;
    \item High dimensionality of the dataset, which makes it more difficult to separate clusters effectively.
\end{itemize}

These issues are mitigated by Ward's method, which doesn't show a dominant biggest cluster, for
all numbers of clusters tested. The minimum cluster size is also consistently more balanced
across different numbers of clusters.\\


Since Ward's method is based on Squared Error, it's not surprising how its SSE is consistently lower than other linkages, as shown in the second graph of figure~\ref{fig:sil_sse_hierarchical_clust}.
What's more interesting is that Ward's method has the lowest Silhouette scores for all numbers
of clusters tested, which indicates that the clusters are not well separated. This is due to
the fact that Ward's method groups the data which resides in the high-density area
mentioned above into smaller clusters, leading to less well-defined boundaries between them.

% By looking at Ward's statistics, the more interesting numbers of clusters to analyze are 4 and 5.
% The first one corresponds to a big decline in maximum cluster size (from 74 to 44\%),
% with a minimum cluster size of around 11\%; the second one has the same maximum cluster size but a smaller minimum cluster size (around 6\%), but has slightly higher Silhouette score
% (17.1\% against 16.8\% for 4 clusters) and a lower SSE. Figure~\ref{fig:dendrograms_ward} shows
% the dendrograms for the two clusterings.
\subsection{Ward's method}
Figure~\ref{fig:dendrograms_ward} a dendrogram and a scatter plot for a clustering obtained through
Ward's Method. Because of the parameters observed in the previous section, cleaner dendrograms,
as well as consistency with other clustering methods, the hierarchical clustering is cut at 4 clusters.
% Both clusterings are set to have 4 clusters, which in both cases leads to cleaner dendrograms.
\begin{figure}[H]
    \centering
    \begin{subfigure}[b]{0.49\textwidth}
        \centering
        \includegraphics[width=\textwidth]{plots/dendrogram_4.png}
        \caption{Ward's Method Dendrogram}
        \label{fig:dendrogram_ward}
    \end{subfigure}
    \begin{subfigure}[b]{0.49\textwidth}
        \centering
        \includegraphics[width=\textwidth]{plots/scatter_ward.png}
        \caption{Ward's Method Scatter plot}
        \label{fig:scatter_ward}
    \end{subfigure}
    \caption{Ward's method clustering}
    \label{fig:dendrograms_ward}
\end{figure}

% The obtained dendrogram has clearly separated clusters, as shown in figure~\ref{fig:dendrogram_ward}.
% The clusters all form at a similar distance (around 12), meaning the SSE increase is similar for
% all clusters.
% It is interesting to note how the smallest cluster is merged with the others at the last merge; this,
% as can be observed in figure~\ref{fig:scatter_ward}, is the cluster which resides in the less dense area of the dataset.
% The other three clusters are closer to each other, as they are a result of the split of the higher density
% area of the dataset. Compared to K-Means, the clusters are not as cleanly separated in the PCA space,
% leading to overlapping areas between them.
The dendrogram in Figure~\ref{fig:dendrogram_ward} reveals well-separated clusters, all of which merge at a
similar linkage distance (approximately 12). This indicates that the increase in within-cluster sum of
squares (SSE) is relatively consistent across all cluster merges, as expected from Ward's method.
Notably, the smallest cluster is the last to be merged, which suggests it is more distinct from the others.
As shown in Figure~\ref{fig:scatter_ward}, this cluster lies in a sparser region of the dataset.
In contrast, the remaining three clusters originate from the denser region and are therefore spatially
closer to one another.
Compared to K-Means, the clusters identified by Ward's method appear less clearly separated in the PCA
projection, resulting in some overlap between adjacent groups.



\subsection{Complete Linkage}
Figure~\ref{fig:dendrograms_complete} shows the dendrogram and scatter plot for a clustering obtained
through Complete Linkage.
Since the tendency of this linkage is to merge into a single cluster, the clustering is cut at 4 clusters,
which helps mitigating the issue. While selecting five clusters would have introduced an additional split
within the largest cluster with similar SSE and Silhouette scores, the resulting group was found to be
poorly separated and lacked meaningful distinction.
As such, it was not considered a valuable contribution to the overall clustering structure.\\

As shown in the dendrogram in Figure~\ref{fig:dendrogram_complete}, the clusters merge at similar linkage
distances (approximately between 1.2 and 1.4), indicating that the maximum within-cluster distances are
comparable across clusters. With respect to the clustering obtained through Ward's method, the clusters
have clearer boundaries along the PCA axes, as the denser area of the dataset is not split into multiple
clusters.\\

It is also interesting to observe how the dendrogram structure differs from that of Ward's method.
In the previous case, the smallest cluster was the last to be merged, reflecting its distinctiveness in
terms of within-cluster variance. In contrast, the Complete Linkage dendrogram shows the two smaller
clusters being merged before the root.
% This suggests that, under this criterion, their relative
% proximity in terms of maximum inter-point distance makes them more similar to each other than to the
% remaining data.
\begin{figure}[H]
    \centering
    \begin{subfigure}[b]{0.49\textwidth}
        \centering
        \includegraphics[width=\textwidth]{plots/dendrogram_complete.png}
        \caption{Complete Linkage Dendrogram}
        \label{fig:dendrogram_complete}
    \end{subfigure}
    \begin{subfigure}[b]{0.49\textwidth}
        \centering
        \includegraphics[width=\textwidth]{plots/scatter_complete.png}
        \caption{Complete Linkage Scatter plot}
        \label{fig:pairplot_ward_5}
    \end{subfigure}
    \caption{Dendrograms for hierarchical clustering with Complete Linkage}
    \label{fig:dendrograms_complete}
\end{figure}



% \begin{minipage}{0.4\textwidth}
%     Figure~\ref{fig:pairplot_ward} shows Hierarchical clustering through Ward's method applied to the dataset, visualized using PCA.
%     The plot shows how the higher density area is split among clusters 1, 2 and 4, while
%     cluster 3 has an overall lower density.
% \end{minipage}
% \hfill
% \begin{minipage}{0.55\textwidth}
%     \centering
%     \includegraphics[width=\textwidth]{plots/pairplot_hierarchical.png}
%     \captionsetup{width=0.9\linewidth, justification=centering}
%     \captionof{figure}{PCA visualization for hierarchical clustering with Ward's method}
%     \label{fig:pairplot_ward}
% \end{minipage}


% k = 4:
% Cluster 1: SSE = 340.29
% Cluster 2: SSE = 282.12
% Cluster 3: SSE = 464.36
% Cluster 4: SSE = 426.15

% k = 5:
% Cluster 1: SSE = 340.29
% Cluster 2: SSE = 66.31
% Cluster 3: SSE = 119.49
% Cluster 4: SSE = 464.36
% Cluster 5: SSE = 426.15

\section{General considerations}\label{sec:considerations} %clustering


\chapter{Classification}
\label{ch:capitolo3}
Classification was performed on the available training set using three different algorithms: K-NN (\textit{K-Nearest Neighbours}), Naïve Bayes and Decision Trees.
For K-NN \textbf{and Naïve Bayes}, a portion of the training set (referred to as the validation set) was used to select the best hyperparameters \textbf{each} model.
The features used in K-NN and Naïve Bayes were normalized, as these models are sensitive to unscaled values.
In particular, a log-transformation and \textbf{SCRIVERE SE StandardScaler O MINMAX} were applied to data.
After training, the models were evaluated on the test set using standard performance metrics. 
The target variables chosen for this task are 2: \texttt{titleType}, and \texttt{has\_LowEngagement}.
These will be discussed in more detail in the corresponding sections below.

\section{Binary classification}\label{sec:binary_classification}
The binary target variable used in this task, \texttt{has\_LowEngagement}, was specifically defined for this purpose. 
It identifies records where the \texttt{numVotes} attribute is less than 100.\\

An analysis of semantically related features was run, in order to decide whether to discard any other feature.
\texttt{userReviewsTotal} showed a 75\% correlation with
\texttt{numVotes}, while \texttt{criticReviewsTotal} has 67\% correlation.
Despite the similarity in correlation
values, \texttt{userReviewsTotal} was deemed too semantically similar to \texttt{numVotes}, whereas
\texttt{criticReviewsTotal} was considered to provide distinct and complementary information.
The correlation value of the second considered not sufficiently high to make the problem trivial.\\

An important aspect of the chosen binary classification task is the class imbalance, with 10287 records
classified as \textit{Low Engagement} and 4668 as \textit{High Engagement} in the training set.
This imbalance was taken into account during model training and evaluation, with a focus on
macro-averaged F1-score to mitigate its impact on the results.\\

\subsection{K-NN}


\subsection{Naïve Bayes}


\subsection{Decision Trees}
For explainability purposes, features were not normalized nor transformed for the Decision Tree model,
as it does not require such preprocessing because it's not based on distance measures, but rather on
decision thresholds.\\

To identify the optimal hyperparameters, a Randomized Search was performed
using Repeated Stratified 5-Fold Cross-Validation with 10 repeats on the training set, optimized for
the macro-averaged F1-score.
The best configuration found used Gini index as the splitting criterion,
a maximum tree depth of 26, and a minimum of 3 samples per leaf.
Post-pruning did not yield any performance improvement and was therefore not applied.
The obtained decision tree is shown in figure~\ref{fig:binary_dt}.
\begin{figure}[H]
    \centering
    \includegraphics[width=0.8\linewidth]{plots/binary_dt.png}
    \captionsetup{justification=centering, width=0.9\linewidth}
    \caption{Decision Tree for binary classification}
    \label{fig:binary_dt}
\end{figure}

Unsurprisingly, the most important feature for the model was \texttt{criticReviewsTotal},
which amounted to 0.6 in the feature importance ranking. The following other 3 more important
features were \texttt{totalCredits} (0.15), \texttt{totalMedia} (0.10) and \texttt{numRegions} (0.09).
These four features take up around 93\% of the total feature importance, and are all present in the first
two splits shown in the Decision Tree.\\
% Classification performance is summarized in Table~\ref{tab:binary_classification_report}.

% \begin{table}[H]
%     \centering
%     \begin{tabular}{lcccc}
%         \toprule
%         \bf{Class} & \bf{Precision} & \bf{Recall} & \bf{F1-score} & \bf{Support} \\
%         \midrule
%         \bf{Low engagement} & 0.86 & 0.90 & 0.88 & 3416 \\
%         \bf{High engagement} & 0.75 & 0.69 & 0.72 & 1561 \\
%         \midrule
%         \bf{Macro avg} & 0.81 & 0.79 & 0.80 & \\
%         \bf{Weighted avg} & 0.83 & 0.83 & 0.83 & \\
%         \midrule
%         % & & \textbf{Train} & \textbf{Test} & \\
%         % \midrule
%         \bf{ROC AUC} & & & 0.87 & \\
%         \bf{Accuracy}  &  & & 0.83 & \\
%         \bottomrule
%     \end{tabular}
%     \caption{Classification report for binary classification}
%     \label{tab:binary_classification_report}
% \end{table}
Train performance was overall similar to the test performance; in particular, the respective accuracies
were of 0.83 and 0.84, and macro-F1 scores were 0.82 and 0.80.
The \textit{High Engagement} class showed low Recall values (0.71 on train set, 0.68 on test set).
This might be a consequence of class imbalance, as well as poor separability of the two classes.
This assumption is further supported by the
Precision scores of the class (0.78 on train, 0.75 on test).

% \begin{figure}[H]
%     \begin{minipage}{0.58\textwidth}
        % Figure~\ref{fig:conf_matr_binary_dt} shows the confusion matrix for the obtained
        % Decision Tree, with results regarding the test set.
        % As can be seen, a significant number of \textit{High Engagement} records was misclassified,
        % leading to a low Recall for that class.
        % This might be a consequence of class imbalance, as well as the possible presence
        % of noise in the data. It's also possible that the two classes are not well separated, leading
        % the model to prioritize the predominant class.\\
%     \end{minipage}
%     \hfill
%     \begin{minipage}{0.38\textwidth}
%         \includegraphics[width=\linewidth]{plots/binary_dt_confusion_matrix.png}
%         \captionsetup{justification=centering, width=0.9\linewidth}
%         \caption{Confusion matrix for binary classification}
%         \label{fig:conf_matr_binary_dt}
%     \end{minipage}
% \end{figure}

\subsection{Model Comparison}

\section{Multiclass classification}\label{sec:multiclass_classification}
Among the multiclass features in the training set, \texttt{titleType} was selected as the target variable
for this task, due to its relevance within the dataset. Because of their strong correlation with
\texttt{titleType}, the features \texttt{canHaveEpisodes} and \texttt{is\_Short} were excluded from the
feature set. Furthermore, since the primary imputation method for missing values in \texttt{runtimeMinutes}
relied on information from the target variable, these values were re-imputed to avoid data leakage.
Specifically, missing entries were filled by sampling from the overall distribution of
\texttt{runtimeMinutes}, without referencing \texttt{titleType}.\\

One final point to note is the imbalance in the target feature (previously shown in
figure~\ref{fig:titleType_distrib}), which was explicitly taken into account
during the design of the models. As for the binary classification task, macro-averaged F1-score was
a key metric for model evaluation, as it provides a balance between each class's precision and recall.



% This feature was created to impute the missing values of the original \texttt{runtimeMinutes} variable,
% but without using the median value according to the titleType. Instead, the missing values were imputed using the help of two variables: \texttt{canHaveEpisodes} and \texttt{is\_Short}
% (as one of the resulting variables of the multi-label one-hot encoding process of the \texttt{genres} attribute).
% In particular, 
% \textbf{SCRIVERE COME E' STATA IMPUTATA NO\_TT - con canhaveepisodes e is\_short preso dai generi}.
% This approach prevents a significant error, as it would be methodologically incorrect to use \texttt{titleType}-based 
% imputation for an attribute when \texttt{titleType} itself is the target variable to predict.
\subsection{K-NN}
\subsection{Naïve Bayes}
\subsection{Decision Trees}
% \begin{table}[H]
%     \centering
%     \begin{tabular}{lcccc}
%         \toprule
%         \bf{Class} & \bf{Precision} & \bf{Recall} & \bf{F1-score} & \bf{Support} \\
%         \midrule
%         \bf{movie}         & 0.85 & 0.88 & 0.87 & 1877 \\
%         \bf{short}         & 0.92 & 0.94 & 0.93 & 766 \\
%         \bf{tvEpisode}     & 0.89 & 0.92 & 0.90 & 1599 \\
%         \bf{tvMiniSeries}  & 0.51 & 0.35 & 0.41 & 81 \\
%         \bf{tvMovie}       & 0.36 & 0.29 & 0.32 & 299 \\
%         \bf{tvSeries}      & 0.89 & 0.94 & 0.91 & 447 \\
%         \bf{tvShort}       & 0.00 & 0.00 & 0.00 & 16 \\
%         \bf{tvSpecial}     & 0.32 & 0.12 & 0.18 & 49 \\
%         \bf{video}         & 0.55 & 0.46 & 0.50 & 250 \\
%         \midrule
%         \bf{Macro avg}     & 0.59 & 0.54 & 0.56 & 5384 \\
%         \bf{Weighted avg}  & 0.82 & 0.84 & 0.83 & 5384 \\
%         \midrule
%         \bf{Accuracy}      &      &      & 0.84 & 5384 \\
%         \bottomrule
%     \end{tabular}
%     \caption{Classification report for multiclass classification (\texttt{titleType})}
%     \label{tab:multiclass_classification_report}
% \end{table}




\subsection{Model Comparison}
Because of class imbalance, for evaluation purposes,
macro-averaged F1-score was heavily considered. %classification

\chapter{Regression}
\label{ch:capitolo4}


\section{Univariate Regression}

\section{Multiple Regression}

\section{Multivariate Regression}
 %regression

\chapter{Pattern Mining}
\label{ch:capitolo2}
The chosen pattern mining technique was Apriori. To perform this task, continuous attributes were discretized according to their distributions. 
The objective of this process was to identify bins that were both semantically meaningful and sufficiently balanced in terms of number of records per bin. 
Among all the available numerical attributes, the ones chosen for the pattern mining task are the following (with their corresponding binning):
\textbf{Aggiustare var sotto quando avremo var definitive }
\begin{itemize}
    \item \texttt{runtimeMinutes\_TitleType}: VeryLowRT (3-30), LowRT (31-60), MediumRT (61-90), HighRT (91-170) 
    \item \texttt{numVotes}: VeryLowNV (5-15), LowNV (16-50), MediumNV (51-150), HighNV (151-997), VeryHighNV (1001-966565) 
    \item \texttt{rating}: VeryLowR (1-3), LowR (4-6), MediumR (7), HighR (8), VeryHighR (9-10)
    \item \texttt{userReviewsTotal}: NoUR (0), FewUR (1-3), ManyUR (4-30), VeryManyUR (31-5727)
    \item \texttt{totalCredits}: VeryLowC (0-15), LowC (16-35), MediumC (36-65), HighC (66-200), VeryHighC (201-15742)
    \item \texttt{criticReviewsTotal}: NoCR(0), FewCR(1), ModerateCR(2-5), ManyCR(6-20), VeryHighCR (21-45), ExtremelyHighCR (46-533)
\end{itemize}

Regarding discrete attributes, \texttt{titleType} was considered for this task but being kept as it is. 
On the other hand, for \texttt{countryOfOrigin\_[continent code]}\footnote{see subsection 1.1.3 for the list of the 7 features} it has been decided to create one variable for each continent, i.e. \texttt{from\_[continent name]}, which has been binarized into:
\begin{itemize}
    \item \texttt{not\_from\_[continent name]}: when the value of the attribute is 0
    \item \texttt{is\_from\_[continent name]}: when the value of the attribute is >= 1
\end{itemize}

The data on which this task was performed was not normalized. 

\section{Extraction of frequent patterns}\label{sec:freq_patterns}


\section{Extraction of rules}\label{sec:rules}



\section{Exploiting rules for target prediction}\label{sec:prediction_rules}
 %pattern mining


% \thispagestyle{empty}
\clearpage

%\listoffigures secondo me non utile
% \thispagestyle{empty}

\printbibliography
%\renewcommand{\listfigurename}{List of figures}
%\listoffigures

\end{document}
