\chapter{Regression}
\label{ch:capitolo4}
Different regression techniques were applied to the dataset, testing
on different combinations of attributes. The target variable chosen for
Univariate and Multiple regression was \texttt{criticReviewsTotal}, while
for Multivariate regression, the target variables were both
\texttt{userReviewsTotal} and \texttt{criticReviewsTotal}.
These were chosen because they offer important insights into the engagement
that a product can generate, which also was the focus of the binary
classification task in section~\ref{sec:binary_classification}.


\section{Univariate and Multiple Regression}
For Univariate Regression, the attribute
\texttt{criticReviewsTotal} was chosen
as the target variable. Aside from the semantic meaning, this choice was also made
because it has a high correlation
with the attribute \texttt{userReviewsTotal}, allowing univariate regression
to be performed, while maintaining a clear separate semantic meaning.
Table~\ref{tab:uni_multi_regression_report} shows the results of the Univariate and Multiple Regression
tasks.
% Figure~\ref{fig:univariate_regression} shows the results of the Univariate Regression task for
% all models but KNN, which doesn't provide an interesting visualization.
% \begin{figure}[H]
%     \centering
%     \begin{subfigure}{0.2565\textwidth}
%         \centering
%         \includegraphics[width=1\textwidth]{plots/linear.png}
%         \caption{Linear regressor}
%         \captionsetup{width=0.9\linewidth, justification=centering}
%         \label{fig:linear}
%     \end{subfigure}
%     \begin{subfigure}{0.24\textwidth}
%         \centering
%         \includegraphics[width=1\textwidth]{plots/ridge.png}
%         \caption{Ridge regressor}
%         \captionsetup{width=0.9\linewidth, justification=centering}
%         \label{fig:ridge}
%     \end{subfigure}
%     \begin{subfigure}{0.24\textwidth}
%         \centering
%         \includegraphics[width=1\textwidth]{plots/lasso.png}
%         \caption{Lasso regressor}
%         \captionsetup{width=0.9\linewidth, justification=centering}
%         \label{fig:lasso}
%     \end{subfigure}
%     \begin{subfigure}{0.24\textwidth}
%         \centering
%         \includegraphics[width=1\textwidth]{plots/dt_regressor.png}
%         \caption{DT regressor}
%         \captionsetup{width=0.9\linewidth, justification=centering}
%         \label{fig:elastic}
%     \end{subfigure}
%     \caption{Univariate regression prediction results in Logarithmic space}
%     \label{fig:univariate_regression}
% \end{figure}

\begin{table}[H]
    \centering
    \begin{tabular}{lccccccc}
        \toprule
        % mae, mse normalized over target variable range
         & \textbf{Intercept} & \textbf{Coefficient} & \textbf{R$^2$} & \textbf{MAE} & \textbf{MSE} & \textbf{$\alpha$} \\
        \midrule
        \textbf{Univariate} \\
        \midrule
        Linear & 3.77 * 10$^-18$ & 0.598 & 0.466 & 0.260 & 1.096 & - \\ % train mae: 0.6427363729986589
        Ridge & 3.77 * 10$^-18$ & 0.598 & 0.466 & 0.260 & 1.096 & 0.1 \\ % alpha=100
        Lasso & 4.41 * 10$^-18$ & 0.498 & 0.412 & 0.272 & 1.206 & 0.1 \\ % alpha=100
        DT & - & - & 0.686 & 0.177 & 0.644 & - \\
        24-NN & - & - & 0.643 & 0.169 & 0.731 & - \\
        \midrule
        \textbf{Multiple}\\
        \midrule
        Linear & - & - & 0.649 & 0.006 & 0.220 & - \\
        Ridge & - & - & 0.649 & 0.006 & 0.220 & 0.1 \\ % alpha=0.1
        Lasso & - & - & 0.627 & 0.005 & 0.233 & 0.1 \\ % alpha=0.1
        DT & - & - & 0.705 & 0.004 & 0.185 & - \\
        7-NN & - & - & 0.770 & 0.003 & 0.176 & - \\
        \bottomrule
    \end{tabular}
    \caption{Classification report for binary classification}
    \label{tab:uni_multi_regression_report}
\end{table}

MSE and MAE are normalized over the range of the target variable; all metrics are calculated on
the test set.\\
As can be seen by the $\alpha$ values, the Ridge and Lasso regressors had their best performances
with high regularization; the Ridge regressor in particular had the same results as the Linear regressor
in the Univariate Regression task, while the Lasso regressor had some differences, with a lower MAE,
but a higher MSE and lower R$^2$ score. The best performing model for Univariate Regression was 
the Decision Tree regressor, and KNN (with 24 neighbors) had similar results, surpassing
Linear, Ridge and Lasso regressors. Both models were tuned through Grid Search Cross-Validation.
The obtained Decision Tree regressor had a maximum depth of 18, with an $\alpha$ of 0.075, and a
minimum impurity decrease of 0.056.




% \begin{table}[H]
%     \centering
%     \begin{tabular}{lccccccccc}
%         \toprule
%         % mae, mse normalized over target variable range
%          & \textbf{Intercept} & \textbf{Coefficient} & \textbf{MAE} & \textbf{MSE} & \textbf{R$^2$} & \textbf{test MAE} & \textbf{test MSE} & \textbf{$\alpha$} \\
%         \midrule
%         \textbf{Univariate} & & & & & \\
%         \midrule
%         Linear & 1.760 & 0.126 & 0.007 & 0.253 & 0.357 & 0.007 & 0.334 & - \\
%         Ridge & 1.760 & 0.126 & 0.007 & 0.253 & 0.357 & 0.007 & 0.334 & 100 \\ % alpha=100
%         Lasso & 1.924 & 0.100 & 0.007 & 0.259 & 0.342 & 0.006 & 0.376 & 100 \\ % alpha=100
%         DT & - & - & 0.005 & 0.127 & 0.678 & 0.004 & 0.194 & - \\
%         24-NN & - & - & 0.004 & 0.135 & 0.658 & 0.004 & 0.223 & - \\
%         \midrule
%         \textbf{Multiple} & & & & & \\
%         \midrule
%         Linear & - & - & 0.006 & 0.157 & 0.601 & 0.006 & 0.220 & - \\
%         Ridge & - & - & 0.006 & 0.157 & 0.601 & 0.006 & 0.220 & 0.1 \\ % alpha=0.1
%         Lasso & - & - & 0.006 & 0.161 & 0.590 & 0.005 & 0.241 & 1 \\ % alpha=1
%         DT & - & - & 0.004 & 0.085 & 0.784 & 0.004 & 0.196 & - \\
%         7-NN & - & - & $7 * 10^{-5}$ & $8 * 10^{-5}$ & $\approx 1$ & 0.003 & 0.176 & - \\
%         \bottomrule
%     \end{tabular}
%     \caption{Classification report for binary classification}
%     \label{tab:binary_classification_report}
% \end{table}
While Linear, Ridge and Lasso's predictions are similar in both 

For Multiple Regression, the target variable was kept the same, in order to
allow for a direct comparison of the results obtained with the two techniques.


\section{Multivariate Regression}
