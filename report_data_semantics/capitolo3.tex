\chapter{Classification}
\label{ch:capitolo3}
Classification was performed on the available training set using three different algorithms: K-NN (\textit{K-Nearest Neighbours}), Naïve Bayes and Decision Trees.
For K-NN \textbf{and Naïve Bayes}, a portion of the training set (referred to as the validation set) was used to select the best hyperparameters \textbf{each} model.
The features used in K-NN and Naïve Bayes were normalized, as these models are sensitive to unscaled values.
After training, the models were evaluated on the test set using standard performance metrics. 
The target variables chosen for this task are 2: \texttt{titleType}, and \texttt{has\_LowEngagement}.
These will be discussed in more detail in the corresponding sections below.

\section{Multiclass classification}\label{sec:multiclass_classification}
Among the multiclass features of the training set, \texttt{titleType} 
was chosen as the target variable for this task due to its relevance in the dataset.\\
An important aspect to highlight is the usage of \texttt{fill\_runtimeMinutes\_notitleType}
as one of the variables to train the models. 
This feature was created to impute the missing values of the original \texttt{runtimeMinutes} variable,
but without using the median value according to the titleType. Instead, the missing values were imputed using \textbf{SCRIVERE COME E' STATA IMPUTATA NO\_TT}.
This approach prevents a significant error, as it would be methodologically incorrect to use \texttt{titleType}-based 
imputation when \texttt{titleType} itself is the target variable to predict.
\subsection{K-NN}
\subsection{Naïve Bayes}
\subsection{Decision Trees}



\section{Binary classification}\label{sec:binary_classification}
The binary target variable used in this task, \texttt{has\_LowEngagement}, was specifically defined for this purpose. 
It identifies records where the \texttt{numVotes} attribute is less than 100.
\subsection{K-NN}
\subsection{Naïve Bayes}
\subsection{Decision Trees}