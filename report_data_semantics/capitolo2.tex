\chapter{Clustering}
\label{ch:capitolo2}

This chapter of the report aims at illustrating the clustering analysis performed on the dataset at hand.
The employed clustering techniques are K-means (Centroid-based), DBSCAN (density-based) and hierarchical clustering.
The analysis conducted using these methods focused on the numeric attributes of the dataset \textbf{CAPIRE SE FARE ANALISI SOLO SU NUMERICHE O INCLUDERE ANCHE BINARIE CON MIXED DISTANCES
MATRIX}, appropriately log-transformed (as illustrated in the \textit{Variable Transformation} section) and normalized with \textbf{STANDARDSCALER o MINMAX???}.
To provide a proper visualization of the results, a Principal Component Analysis has been conducted on the preprocessed data. 
In particular, the optimal number of components that has been found is \textbf{se usiamo solo numeriche dovrebbe essere 4/5} (as in \textbf{AGGIUNGERE FIGURA}).


\section{Centroid-based methods}\label{sec:centroid_based}


\section{Density-based methods}\label{sec:density_based}


\section{Analysis by hierarchical clustering}\label{sec:hierarchical}


\section{General considerations}\label{sec:considerations}